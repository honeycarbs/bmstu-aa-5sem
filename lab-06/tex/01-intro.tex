\setcounter{page}{2}
\addchap{Введение}

Существует довольно обширный класс задач оптимизации, для которых нет быстрых полиномиальных алгоритмов определения решений. Для таких задач имеет смысл использовать эвристические алгоритмы\cite{ant-cause}, которые не гарантируют оптимальных решений, однако, при правильной параметризации такие алгоритмы обеспечивают наличие решений наилучшего, насколько это возможно, качества.

На сегодняшний день большую популярность имеют методы, основывающиеся на природных механизмах. Такие алгоритмы позволяют решать множество сложных оптимизационных задач, таких как задачи полихромии графов, задачи о распределении и некоторые транспортные задачи. Алгоритм, имитирующий поведение муравьиной колонии, входит в класс алгоритмов <<роевого интеллекта>>\cite{colony}. 

В представленной работе будет предложен метод оптимизации задачи коммивояжера, использующий алгоритм муравьиной колонии. Целью данной работы является анализ предложенной оптимизации и параметризация алгоритма для двух классов данных. Для достижения поставленной цели необходимо выполнить следующие задачи:
\begin{itemize}
	\item реализовать алгоритм полного перебора и алгоритм муравьиной колонии;
	\item разработать два класса эквивалентности для параметризации алгоритма муравьиной колонии;
	\item провести параметризацию для разработанных классов эквивалентности;
	\item сравнить алгоритмы, согласно следующим параметрам:
	\begin{itemize}
		\item качество полученного решения;
		\item временные затраты на получение решения.
	\end{itemize}
\end{itemize}
Сравнительный анализ алгоритмов будет оформлен в виде таблиц, из которых можно будет сделать вывод об эффективности предложенной в работе оптимизации решения задачи коммивояжера.