\chapter{Технологический раздел}\label{sec:impl}
Раздел содержит требования к реализованному программному обеспечению, листинги кода и результаты рестирования программного обеспечения. 
\section{Требования к ПО}

Программное обеспечение должно удовлетворять следующим требованиям:
\begin{itemize}
	\item программа получает на вход размер очереди;
	\item программа выдает журналирование работы конвейера;
	\item генератор подстрок выдает ненулевые строки, состоящие из строчных и заглавных символов;
	\item подстрока, которую выдает генератор подстрок, должна содержаться в исходной строке.
\end{itemize}

\section{Средства реализации} 

Для реализации ПО был выбран компилируемый многопоточный язык программирования Golang, поскольку язык отличается легкой и быстрой сборкой программ и автоматическим управлением памяти.
В качестве среды разработки была выбрана среда VS Code, написание сценариев осуществлялось утилитой make.

\section{Листинги кода}
\subsection{Структуры и типы}
Заявка в очереди имеет следующую структуру: (\ref{lst:type-pipe})
\captionsetup{singlelinecheck = false, justification=raggedright}
\begin{lstlisting}[label=lst:type-pipe,caption=Структура заявки в очереди]
type PipeTask struct {
	num  			int
	generated		bool
	skip_table_made bool
	pattern_mached	bool
	
	start_generating time.Time
	end_generatig    time.Time
	start_table		 time.Time
	end_table		 time.Time
	start_match		 time.Time
	end_match		 time.Time

	source        []rune
	pattern       []rune
	skip_table    map[rune]int
	pattern_index int
}
\end{lstlisting}
Структура очереди продемонстрирована на листинге \ref{lst:queue-type}:
\begin{lstlisting}[label=lst:queue-type,caption=Структура очереди]
type Queue struct {
	queue [](*PipeTask)
	size  int
}
\end{lstlisting}
\subsection{Очередь}
Реализация очереди продемонстрирована на листинге \ref{lst:queue}.
\begin{lstlisting}[label=lst:queue,caption=Очередь]
func initQueue(size int) *Queue {
	qu := new(Queue)
	qu.queue = make([](*PipeTask), size)
	qu.size = -1
	
	return qu
}

func (qu *Queue) enqueue (t *PipeTask) {
	if (qu.size != len(qu.queue) - 1) {
		qu.queue[qu.size + 1] = t
		qu.size++
	}
}

func (qu *Queue) dequeue () *PipeTask {
	t := qu.queue[0]
	qu.queue = qu.queue[1:]
	qu.size--
	return t
}
\end{lstlisting}
\subsection{Реализация параллельного конвейера}
Реализация параллельного конвейера продемонстрирована на листинге \ref{lst:pipeline}.
\begin{lstlisting}[label=lst:pipeline,caption=Реализация параллельного конвейера]
func Pipeline(count int, ch chan int) *Queue {
	first := make(chan *PipeTask, count)
	second := make(chan *PipeTask, count)
	third := make(chan *PipeTask, count)
	
	line := initQueue(count)
	
	gen_string := func() {
		for {
			select {
				case pipe_task := <-first:
				pipe_task.generated = true
				
				pipe_task.start_generating = time.Now()
				
				pipe_task.source = bmsearch.GenerateRune(STRING_SIZE)
				pipe_task.pattern = bmsearch.GeneratePattern(
									pipe_task.source, PATTERN_SIZE)
				
				pipe_task.end_generatig = time.Now()
				
				second <- pipe_task
			}
		}
	}
	
	get_table := func() {
		for {
			select {
				case pipe_task := <-second:
				pipe_task.skip_table_made = true
				
				pipe_task.start_table = time.Now()
				
				pipe_task.skip_table = bmsearch.ConstructSkipTable(
									   pipe_task.pattern)
				pipe_task.end_table = time.Now()
				
				third <- pipe_task
			}
		}
	}
	
	match := func() {
		for {
			select {
				case pipe_task := <-third:
				pipe_task.pattern_mached = true
				
				pipe_task.start_match = time.Now()
				pipe_task.pattern_index = bmsearch.FindFirstIndex(
												pipe_task.source, 
												pipe_task.pattern,
												pipe_task.skip_table)
				pipe_task.end_match = time.Now()
				
				line.enqueue(pipe_task)
				if (pipe_task.num == count - 1) {
					ch <- 0
				}
			}
		}
	}
	go gen_string()
	go get_table()
	go match()
	
	
	for i := 0; i < count; i++ {
		pipe_task := new(PipeTask)
		pipe_task.num = i + 1
		first <- pipe_task
	}
	return line
}
\end{lstlisting}
\subsection{Реализация синхронного конвейера}
Процедуры, необходимые для реализации синхронного конвейера представлены на листингах \ref{lst:sync-string} --  \ref{lst:sync-match}.
\begin{lstlisting}[label=lst:sync-string,caption=Получение подстрок]
func gen_string_sync(task *PipeTask) *PipeTask {
	task.generated = true
	
	task.start_generating = time.Now()
	
	task.source = bmsearch.GenerateRune(STRING_SIZE)
	task.pattern = bmsearch.GeneratePattern(task.source, PATTERN_SIZE)
	
	task.end_generatig = time.Now()
	
	return task
}
\end{lstlisting}
\begin{lstlisting}[label=lst:sync-table,caption=Получение словаря]
func get_table_sync(task *PipeTask) *PipeTask {
	task.skip_table_made = true
	
	task.start_table = time.Now()
	task.skip_table = bmsearch.ConstructSkipTable(task.pattern)
	task.end_table = time.Now()
	
	return task
}
\end{lstlisting}

\begin{lstlisting}[label=lst:sync-match,caption=Поиск подстроки]
func match_sync(task *PipeTask) *PipeTask {
	task.pattern_mached = true
	
	task.start_match = time.Now()
	task.pattern_index = bmsearch.FindFirstIndex(task.source, 
								  task.pattern, task.skip_table)
	task.end_match = time.Now()
	
	return task
}
\end{lstlisting}
На листинге \ref{lst:sync} представлена реализация синхронного конвейера.
\begin{lstlisting}[label=lst:sync,caption=Синхронный конвейер]
func Sync(count int) *Queue {
	line_first := initQueue(count)
	line_second := initQueue(count)
	line_third := initQueue(count)
	
	for i := 0; i < count; i++ {
		pipe_task := new(PipeTask)
		pipe_task = gen_string_sync(pipe_task)
		line_first.enqueue(pipe_task)
		if (len(line_first.queue) != 0) {
			pipe_task = get_table_sync(line_first.dequeue())
			line_second.enqueue(pipe_task)
			if (len(line_second.queue) != 0) {
				pipe_task = match_sync(line_second.dequeue())
				line_third.enqueue(pipe_task)
			}
		}
	}
	return line_third
}
\end{lstlisting}

\subsection{Поиск подстроки}
Реализация получения словаря вида символ - сдвиг представлена на листинге \ref{lst:table}.
\begin{lstlisting}[label=lst:table,caption=Получение словаря вида символ - сдвиг]
func ConstructSkipTable(pattern []rune) map[rune]int {
	length := len(pattern)
	skips := make(map[rune]int)
	for i, char := range pattern {
		if i < length - 1 {
			skips[char] = length - i - 1
		} else {
			skips[char] = length
		}
	}
	return skips
}
\end{lstlisting}

Генерация строк и подстрок представлена на листинге \ref{lst:generate}. Реализация алгоритма Бойера -- Мура -- Хорспула представлена на листинге \ref{lst:search}.
\begin{lstlisting}[label=lst:generate,caption=Генерация строк и подстрок]
	func GenerateRune(size int) []rune {
		var letters = []rune("abcdefghijklmnopqrstuvwxyz
							  ABCDEFGHIJKLMNOPQRSTUVWXYZ")
		rand.Seed(time.Now().UnixNano())
		b := make([]rune, size)
		for i := range b {
			b[i] = letters[rand.Intn(len(letters))]
		}
		return b
	}
	func GeneratePattern(src []rune, size int) []rune {
		rand.Seed(time.Now().UnixNano())
		start := rand.Intn(len(src) - size)
		length := len(src)
		
		return src[start : length - 1]
	}
\end{lstlisting}

\begin{lstlisting}[label=lst:search,caption=Алгоритм Бойера -- Мура -- Хорспула]
func FindFirstIndex(source, pattern []rune, skips map[rune]int) int {
	len_ptrn := len(pattern)
	len_src := len(source)
	
	j := len_ptrn - 1
	for j < len_src {
		k := j
		i := len_ptrn - 1
		for i >= 0 {
			if source[k] != pattern[i] {
				if skip, found := skips[source[j]]; found {
					j += skip
				} else {
					j += len_ptrn
				}
				break
			}
			k--
			i--
		}
		if i < 0 {
			return k + 1
		}
	}
	return -1
}
\end{lstlisting}

\section{Тестирование ПО}
Тестирование ПО проводилось написанием журналирования для вывода содержания заявок очереди(\ref{lst:testing}).
\begin{lstlisting}[label=lst:testing,caption=Тестирование ПО]
func print_map(m map[rune]int) {
	fmt.Print("( ")
	for i, ch := range m {
		fmt.Printf("%c:%v ", i, ch)
	}
	fmt.Print(") \\\\ \n")
}

func Log(qu *Queue) {
	fmt.Println("")
	line := qu.queue
	for i := range line {
		if line[i] != nil {
			fmt.Printf("%3v & %12v & %10v & ", i, 
			string(line[i].source),
			string(line[i].pattern))
			print_map(line[i].skip_table)
			fmt.Printf(" & %v \\\\ \n", line[i].pattern_index)
		}
	}
}
\end{lstlisting}
На очереди размером в 5 элементов программа показала следующие результаты(\ref{table:testing}):
\begin{table}[H]
	\captionsetup{singlelinecheck = false, justification=raggedleft}
	\renewcommand{\arraystretch}{1.8}
	\caption{Тестирование ПО}
	\resizebox{\textwidth}{!}{
	\begin{tabular}{||c|c|c|c|l|c||}
		\hline
		№ & Строка & Подстрока & Присутствие & Словарь & Результат\\ \hline
			1 &   MeHJWZigCQ &  MeHJWZigC & + &  ( W:4 g:1 C:9 e:7 H:6 J:5 M:8 Z:3 i:2  & 0 \\ 
			2 &   quOFEdkUfL &   uOFEdkUf & + &  ( E:4 d:3 k:2 U:1 f:8 u:7 O:6 F:5  & 1 \\ 
			3 &   XKkbxZMRku &     bxZMRk & + &  ( R:1 k:6 b:5 x:4 Z:3 M:2  & 3 \\ 
			4 &   SyHEexBMWH &   yHEexBMW & + &  ( x:3 B:2 M:1 W:8 y:7 H:6 E:5 e:4  & 1 \\ 
			5 &   bXbEVQAUEZ &  bXbEVQAUE & + &  ( E:9 V:4 Q:3 A:2 U:1 b:6 X:7  & 0 \\ \hline
	\end{tabular}
	}
	\label{table:testing}
\end{table}

\section{Вывод}
Было написано и протестировано программное обеспечение для решения поставленной задачи. 