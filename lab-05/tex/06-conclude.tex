\addchap{Заключение}
В работе была теоретически проанализирована организация конвейерная обработка данных. Вычислительный конвейер был реализован на трех потоках и на одном потоке. Алгоритм Бойера -- Мура -- Хорспула был декомпозирован для обработки конвейерными линиями. 
Эксперимент показал, что распараллеливание алгоритма Бойера -- Мура -- Хорспула приводит к выигрышу в 6\%. Выигрыш происходит исключительно за счет обеспечения меньшего простоя очереди и ситуаций, когда простоя нет вовсе. 

Этапы обработки стабильны по времени в пределах погрешности лишь за счет некоторых ограничений, которые были указаны в разделе \ref{sec:design} -- такое решение было принято с целью усреднения времени работы каждой из лент конвейера. 

Следовательно, реализация данного алгоритма на параллельном конвейере хоть и эффективна, но на ограниченном объеме данных. 