\addchap{Заключение}
Электронная вычислительная техника эволюционирует непрерывно и стремительно. Но, даже при таких темпах, возникают ситуации, когда алгоритмы имеют значительный выигрыш только на таких данных, хранение которых невозможно современными ЭВМ. 
Проанализированный алгоритм на небольших данных имеет выигрыш во времени в среднем на 5\%, что связано с меньшим количеством трудоемких операций и препроцессировании части данных. В этом и состоит главная идея алгоритма.  

Алгоритм Копперсмита -- Винограда считался самым быстрым до 2010 года. На декабрь 2020 года самым быстрым алгоритмом считается алгоритм, доказанный математиком Вирджинией Василевской-Вильямс\cite{faster}. Поэтому в настоящее время алгоритм не имеет преимуществ и используется только в теории для доказательства некоторых математических тождеств. 