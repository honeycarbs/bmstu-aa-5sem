\chapter{Исследовательская часть}
\section{Технические характеристики}
Тестирование выполнялось на устройстве со следующими техническими характеристиками:
\begin{itemize}
	\item Операционная система Ubuntu 20.04.1 LTS;
	\item Память 7 GiB
	\item Процессор Intel(R) Core(TM) i3-8145U CPU @ 2.10GHz
\end{itemize}
Во время тестирования устройство было подключено к блоку питания и не нагружено никакими приложениями, кроме встроенных приложений окружения, окружением и системой тестирования.
\section{Время выполнения алгоритмов}
Алгоритмы тестировались при помощи профилирования -- сбора характеристик работы программы: времени выполнения и затрат по памяти. Для каждой функции были написаны "бенчмарки"\cite{test}, представленные встроенными в Golang средствами. Бенчмарки, реализованные стандартными средствами Golang автоматически делают некоторое количество замеров, предоставляя результат с некоторой погрешностью.
\begin{lstlisting}[label=bench,caption=Пример бенчмарка]
func BenchmarkDamerauIterative20(b *testing.B)  {
	src := []rune("zyapjbnxuykcwtnmexid")
	dest := []rune("ickfleanmgcpakskibcx")
	
	for i := 0; i < b.N; i++ {
		dist.DamerauLevenshteinIterative(src, dest)
	}
}
\end{lstlisting}
Результаты тестирования приведены в таблице \ .Прочерк в таблице означает что тестирование для этого набора данных не выполнялось.


\begin{table}[h!]
	\begin{center}
		\caption{Время выполнения алгоритмов}
		\begin{tabular}{ ||p{1.5cm}||p{2cm}|p{3cm}|p{2cm}|p{2cm}|p{3.5cm}||  }
			\hline
			\multirow{2}{*}{Длина}& \multicolumn{5}{c||}{Время выполнения()} \\[1.5ex]
			\cline{2-6} 
			строк&LIter & LRec & LRecMem & DLIter & DLRec \\ [1.5ex] 
			\hline\hline
			5 & 468 & 18762 & 959 & 531 &  22974\\ [0.7ex]
			10 & 1261& 91229882 & 3442 & 1519 & 139913520\\ [0.7ex]
			15 & 2614& 502162161429 & 7243 & 3144 & 981845877565 \\ [0.7ex]
			20 & 4244& - & 12883 & 5309 &  -\\ [0.7ex]
			30 & 8806& - & 28096 & 11257 & - \\ [0.7ex]
			50 & 22944& - & 76094 & 29965 & - \\ [0.7ex]
			100 & 88145& - & 304941 &117059 & - \\ [0.7ex]
			200 & 338736& - & 1267569 & 451253 & - \\ [0.7ex]
			\hline
		\end{tabular}
		\label{time-table}
	\end{center}
\end{table}

\begin{figure}[h!]
	\begin{center}
		\begin{tikzpicture}
			\begin{axis}[
				legend pos = north west,
				xlabel=длина строки,
				ylabel=наносекунды,
				minor tick num = 1,
				grid = both,
				major grid style = {lightgray},
				minor grid style = {lightgray!25},
				xtick distance = 50,
				width = 0.9\textwidth,
				height = 0.7\textwidth]

				\addplot[
				blue,
				semithick,
				mark = x,
				mark size = 3pt,
				thick,
				] file {assets/lev-iter-perf.dat};
				
				\addplot[
				red,
				semithick,
				mark = *,
				] file {assets/d-lev-iter-perf.dat};
				
				\legend{
					Расстояние Левенштейна,
					Расстояние Дамерау -- Левенштейна
				}
			\end{axis}
		\end{tikzpicture}
	\end{center}
\caption{Сравнение расстояния итеративного и матричного расстояния Левенштейна}
\end{figure}

\begin{figure}[h!]
	\begin{center}
		\begin{tikzpicture}
			\begin{axis}[
				legend pos = north west,
				xlabel=длина строки,
				ylabel=наносекунды,
				minor tick num = 1,
				grid = both,
				major grid style = {lightgray},
				minor grid style = {lightgray!25},
				xtick distance = 50,
				width = 0.9\textwidth,
				height = 0.7\textwidth]
				
				\addplot[
				green,
				thick,
				mark = o,
				mark size = 3pt,
				semithick,
				] file {assets/lev-iter-perf.dat};
				
				\addplot[
				magenta,
				semithick,
				mark = x,
				mark size = 3pt
				] file {assets/lev-memo-perf.dat};
				
				\legend{
					итеративный алгоритм,
					мемоизация
				}
			\end{axis}
		\end{tikzpicture}
	\end{center}
	\caption{Сравнение рекурсивного и итеративного матричного расстояния Левенштейна}
\end{figure}

\section{Использование памяти}

Итеративные алгоритмы Левенштейна и Дамерау -- Левенштейна используют одинаковое количество памяти, следовательно, смысла их анализировать нет. Более информативным будет исследование рекурсивных и матричных реализаций этих алгоритмов. Максимальная глубина стека при вызове рекурсивных функций имеет следующий вид:
\begin{equation}\label{rec-mem}
	M_{recursive} = (n \cdot lvar + ret + ret_{int}) \cdot depth
\end{equation}
Где:
\[
\begin{array}{l}
	n\text{ -- количество аллоцированных локальных переменных}; \\
	lvar\text{ -- размер переменной типа int} \\
	ret -- \text{адрес возврата;}\\
	ret_{int} -- \text{возвращаемое значение;}\\
	depth\text{ --  максимальная глубина стека вызова, которая равна } \\
	|S_1| + |S_2|.
\end{array}
\]
Использование памяти при итеративных реализациях: 
\begin{equation}
	M_{iter} = |S_1| + |S_2| + (|S_1| + 1 \cdot |S_2| + 1) \cdot lvar + n \cdot lvar + ret + ret_{int}
\end{equation}
Где $(|S_1| + 1 \cdot |S_2| + 1) \cdot lvar$ -- место в памяти под матрицу расстояний.

\section{Вывод}
Рекурсивный алгоритм Левенштейна работает дольше итеративных реализаций -- время этого алгоритма увеличивается в геометрической прогрессии с ростом размера строк.
Рекурсивный алгоритм с мемоизацией превосходит простой рекурсивный алгоритм по времени. 
Расстояние Дамерау -- Левенштейна по результатом замеров практически не отличается от расстояния Левенштейна. Однако, в системах автоматического исправления текста, где транспозиция встречается чаще, расстояние Дамерау -- Левенштейна будет наиболее эффективным алгоритмом. 
По расходу памяти все реализации проигрывают рекурсивной за счет большого количества выделенной памяти под матрицу расстояний.
Стоит отметить, что для языков, где возможна передача указателя на массивы, самым эффективным и по времени, и по памяти будет алгоритм, использующий мемоизацию.