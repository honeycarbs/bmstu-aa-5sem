\chapter{Заключение}

В рамках лабораторной работы были рассмотрены два алгоритма нахождения редакторского расстояния -- расстояние Левенштейна и расстояние Дамерау -- Левенштейна. Во время аналитического изучения алгоритмов были выявлены смысловые различия между двумя алгоритмами -- расстояние Дамерау -- Левенштейна более эффективно в системах автоматической замены текста, где наиболее часто встречающаяся редакторская операция -- это транспозиция. В других случаях, если алгоритмы работают не с буквами в естественном языке, рациональнее использовать алгоритм расстояние Левенштейна. 
Самая оптимальная реализация по памяти -- рекурсивный алгоритм, самая оптимальная реализация по времени -- итеративный алгоритм, использующий таблицу расстояний. Для языков, где возможна передача указателя на массивы, самым эффективным по времени и по памяти будет алгоритм, использующий мемоизацию. 
В ходе лабораторной работы получены навыки динамического программирования, реализованы изученные алгоритмы.