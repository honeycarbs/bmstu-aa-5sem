\chapter{Введение}

Нахождение редакционного расстояния -- одна из задач компьютерной лингвистики, которая находит применение в огромном количестве областей, начиная от предиктивных систем набора текста и заканчивая разработкой искусственного интеллекта. Впервые задачу поставил советский ученый В. И. Левенштейн \cite{Lev1965}, впоследствии её связали с его именем. В данной работе будут рассмотрены алгоритмы редакционного расстояния Левенштейна и расстояние Дамерау — Левенштейна.

Расстояния Левенштейна -- метрика, измеряющая разность двух строк символов, определяемая в количестве редакторских операций(а именно удаления, вставки и замены), требуемых для преобразования одной последовательности в другую.  Расстояние Дамерау — Левенштейна -- модификация, добавляющая к редакторским операциям транспозицию, или обмен двух соседних символов местами.

Алгоритмы находят применение не только в компьютерной лингвистике (например, при реализации предиктивных систем при вводе текста), но и, например, при работе с утилитой diff и ей подобными. Также у алгоритма существуют более неочевидные применения, где операции проводятся не над буквами в естественном языке. Алгоритм применяется для распознавания текста на нечетких фотографиях. В этом случае сравниваются последовательности черных и белых пикселей на каждой строке изображения.
Нередко алгоритм используется в биоинформатике для определения схожести разных участков ДНК или РНК.

Алгоритмы имеют некоторое количество модификаций, позволяющих эффективнее решать поставленную задачу. В данной работе будут предложены реализации алгоритмов, использующие парадигмы динамического программирования.

Цель лабораторной работы -- получить навыки динамического программирования. 
Задачами лабораторной работы являются изучение и реализация алгоритмов Левенштейна и Дамерау — Левенштейна, применение парадигм динамического программирования при реализации алгоритмов и сравнительный анализ алгоритмов на основе экспериментальных данных.
