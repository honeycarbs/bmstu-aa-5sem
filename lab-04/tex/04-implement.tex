\chapter{Технологический раздел}\label{sec:impl}

\section{Требования к ПО}

Программное обеспечение должно удовлетворять следующим требованиям:
\begin{itemize}
	\item ПО получает на вход текстовый файл с бинарным лабиринтом размером не превышающим $1000 \times 1000$;
	\item ПО предоставляет пользователю введенный лабиринт, в котором обозначено найденное минимальное расстояние.
\end{itemize}

\section{Средства реализации} 
В качестве языка, на котором будет реализовано программное обеспечение, был выбран язык C++. Выбор языка обусловлен тем, что стандартная библиотека языка имеет существующую реализацию очереди\cite{queue++} и реализацию работы с нативными потоками\cite{threads++}.
\section{Листинги кода}
Листинг и \ref{lst:sync-wrap} и \ref{lst:par-wrap} демонстрируют функции -- обертки над функцией, выполняющей непосредственно сам поиск в ширину. 
\captionsetup{singlelinecheck = false, justification=raggedright}
\begin{lstlisting}[label=lst:sync-wrap,caption=функция - обретка синхронного алгоритма Ли]
int leeRootSync(Map &map, const coord_t &src, coord_t &dest) {
	CQueue queue;
	queue.push({src, 0});
	
	bool f = false;
	parallelRoutine(queue, map, dest, f);
	return map.getSlot(dest);
}
\end{lstlisting}

\begin{lstlisting}[label=lst:par-wrap,caption=функция - обретка параллельного алгоритма Ли]
int leeRootParallel(Map &map, const coord_t &src, coord_t &dest,
				    int thread_count) {
	CQueue queue;
	
	queue.push({src, 0});
	std::vector<std::thread> threads;
	bool path_found = false;
	threads.reserve(thread_count);
	for (int i = 0; i < thread_count; i++) {
		threads.emplace_back(
		std::thread(parallelRoutine, std::ref(queue), std::ref(map),
					dest, std::ref(path_found)));
	}
	for (int i = 0; i < thread_count; i++) {
		threads[i].join();
	}
	return map.getSlot(dest);
}
\end{lstlisting}
Функция - обертка над распараллеленным алгоритмом включает в себя создание потоков. Сама функция, выполняющая двоичный поиск, представлена в листинге \ref{lst:bfs}.
\begin{lstlisting}[label=lst:bfs,caption=поиск в ширину]
void parallelRoutine(CQueue &queue, Map &map, coord_t dest,
					 bool &path_found) {
	while (true) {
		if (path_found) {
			if (queue.size() == 0) {
				return;
			}
			path_found = false;
		}
		auto opt = queue.pop();
		if (!opt)
		continue;
		
		auto pair = *opt;
		auto coord = pair.first;
		auto val = pair.second + 1;
		if (pair.first == dest) {
			map.setSlot(pair.first, pair.second);
			continue;
		}
		if (!checkCoord(map, coord, val)) {
			if (queue.size() == 0) {
				path_found = true;
			}
			continue;
		}
		
		map.setSlot(pair.first, pair.second);
		
		std::initializer_list<std::pair<coord_t, int>> list = {
			{{coord.first + 1, coord.second}, val},
			{{coord.first - 1, coord.second}, val},
			{{coord.first, coord.second + 1}, val},
			{{coord.first, coord.second - 1}, val}};
		queue.push(list);
	}
}
\end{lstlisting}

Граф - лабиринт представлен в качестве матрицы, представленной на листингах \ref{lst:matrix} и \ref{lst:matrix-header}.
\begin{lstlisting}[label=lst:matrix-header,caption=Класс матрицы - бинарного лабиринта]
using coord_t = std::pair<int, int>;
using CQueue = PQueue<std::pair<coord_t, int>>;

class Map {
	public:
	const int WALL = -2;
	const int UNSEEN = -1;
	
	explicit Map(std::string filename);
	int getSlot(int x, int y);
	int getSlot(std::pair<int, int> c);
	void setSlot(int x, int y, int state);
	void setSlot(std::pair<int, int> c, int state);
	void outputStdput();
	void clear();
	
	long width() { return width_; };
	long height() { return height_; };
	
	private:
	long width_;
	long height_;
	std::vector<std::vector<int>> slots_;
	std::mutex mutex_;
};
\end{lstlisting}
~\newline

\begin{lstlisting}[label=lst:matrix,caption=Реализация класса матрицы - бинарного лабиринта]
int Map::getSlot(int x, int y) {
	if (x < 0 || y < 0 || x >= width_ || y >= height_)
	return WALL;
	std::lock_guard<std::mutex> guard(mutex_);
	return slots_[x][y];
}
int Map::getSlot(std::pair<int, int> c) {
	return getSlot(c.first, c.second);
}

void Map::setSlot(int x, int y, int state) {
	if (x < 0 || y < 0 || x > width_ || y >= height_)
	throw std::exception();
	
	std::lock_guard<std::mutex> guard(mutex_);
	slots_[x][y] = state;
}

void Map::setSlot(std::pair<int, int> c, int state) {
	this->setSlot(c.first, c.second, state);
}
\end{lstlisting}

Реализация очереди, безопасной для распараллеливания, вводится оберткой очереди в класс и инкапсулированием мьютекса, что видно на листинге \ref{lst:queue}.

 \begin{lstlisting}[label=lst:queue,caption=Шаблонный класс очереди]
template<typename T>
class PQueue {
	public:
	PQueue() = default;
	PQueue(const PQueue<T> &) = delete ;
	PQueue& operator=(const PQueue<T> &) = delete ;
	
	virtual ~PQueue() = default;
	
	unsigned long size() const {
		std::lock_guard<std::mutex> lock(mutex_);
		return queue_.size();
	}
	
	std::optional<T> pop() {
		std::lock_guard<std::mutex> lock(mutex_);
		if (queue_.empty()) {
			return {};
		}
		_i++;
		T tmp;
		if (_i % 2) {
		tmp = queue_.front();
		queue_.pop_front();
	}
	else {
		tmp = queue_.back();
		queue_.pop_back();
	}
	return tmp;
}

void push(const T &item) {
	std::lock_guard<std::mutex> lock(mutex_);
	
	queue_.push_back(item);
}

template<class P>
void push(const P &list) {
	std::lock_guard<std::mutex> lock(mutex_);
	for (auto &item: list) {
		queue_.push_front(item);
	}
}

private:
	int _i = 0;
	std::list<T> queue_;
	mutable std::mutex mutex_;
	
	bool empty() const {
		return queue_.empty();
}
};
 \end{lstlisting}


\captionsetup{singlelinecheck = false, justification=raggedleft}
\section{Тестирование ПО}
\begin{table}[H]
	\renewcommand{\arraystretch}{1.9}
	\caption{Тестовые случаи}
	\resizebox{\textwidth}{!}{
		\begin{tabular}{||c|c|c|c|c|c|c||}
			\hline
			\multirow{2}{*}{№} & \multirow{2}{*}{Матрица} & \multirow{2}{*}{Вход} & \multirow{2}{*}{Выход} & \multirow{2}{*}{Результирующая матрица} & \multicolumn{2}{c||}{Ответ} \\ \cline{6-7} 
			&  &  &  &  & Параллельный & Синхронный \\ \hline\hline
			1 & \begin{tabular}[c]{@{}l@{}}$\begin{matrix}[.5]\\   1 & 0 & 1 & 1 & 1 \\\\   1 & 0 & 0 & 1 & 1 \\\\   1 & 1 & 0 & 0 & 1 \\\\   1 & 0 & 0 & 1 & 1 \\\\   1 & 1 & 0 & 1 & 1\\ \end{matrix}$\end{tabular} & 0 1 & 4 3 & \begin{tabular}[c]{@{}l@{}}$\begin{matrix}[.5]\\     -2 &  0 & -2 & -2 & -2 \\\\     -2 &  1 &  2 & -2 & -2 \\\\     -2 & -2 &  3 &  4 & -2 \\\\     -2 &  5 &  4 & -2 & -2 \\\\     -2 & -2 &  5 &  6 & -2\\ \end{matrix}$\end{tabular} & 6 & 6 \\ \hline
			2 & \begin{tabular}[c]{@{}l@{}}$\begin{matrix}[.5]\\   0 & 0 & 0 & 0 & 0 \\\\   0 & 0 & 0 & 0 & 0 \\\\   0 & 0 & 0 & 0 & 0 \\\\   0 & 0 & 0 & 0 & 0 \\\\   0 & 0 & 0 & 0 & 0\\ \end{matrix}$\end{tabular} & 0 1 & 4 3 & \begin{tabular}[c]{@{}l@{}}$\begin{matrix}[.5]\\     1 & 0 & 1 & 2 & 3 \\\\     2 & 1 & 2 & 3 & 4 \\\\     3 & 2 & 3 & 4 & 5 \\\\     4 & 3 & 4 & 5 & 6 \\\\     5 & 4 & 5 & 6 & 7\\ \end{matrix}$\end{tabular} & 6 & 6 \\ \hline
			3 & 0 & 0 0 & 0 0 & 0 & 0 & 0 \\ \hline
			4 & \begin{tabular}[c]{@{}l@{}}$\begin{matrix}[.5]\\   1 & 1 & 1 & 1 & 1 \\\\   1 & 1 & 1 & 1 & 1 \\\\   1 & 1 & 1 & 1 & 1 \\\\   1 & 1 & 1 & 1 & 1 \\\\   1 & 1 & 1 & 1 & 1\\ \end{matrix}$\end{tabular} & 0 1 & 4 3 & \begin{tabular}[c]{@{}l@{}}$\begin{matrix}[.5]\\ -2 &-2 &-2 &-2 &-2 \\\\ -2 &-2 &-2 &-2 &-2 \\\\ -2 &-2 &-2 &-2 &-2 \\\\ -2 &-2 &-2 &-2 &-2 \\\\ -2 &-2 &-2 &-2 &-2 \\ \end{matrix}$\end{tabular} & 0 & 0 \\ \hline
		\end{tabular}
	}
	\label{table:testing}
\end{table}

\section{Вывод}
Было написано и протестировано программное обеспечение для решения поставленной задачи.
%$\begin{matrix}
%	-2 &  0 & -2 & -2 & -2 \\
%	-2 &  1 &  2 & -2 & -2 \\
%	-2 & -2 &  3 &  4 & -2 \\
%	-2 &  5 &  4 & -2 & -2 \\
%	-2 & -2 &  5 &  6 & -2
%\end{matrix}$